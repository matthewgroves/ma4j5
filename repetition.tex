\documentclass[11pt]{article}
\usepackage{amsmath}
\usepackage{amsfonts}
\begin{document}


\underline{Repetition}: We introduced chemical reactions of the type $P \rightarrow P'$, with P,P' chemical complexes.\\
Then a chemical reaction can be written in standard form as: $$\alpha_{j1}S_{1}+....+ \alpha_{js}S_{s} \rightarrow \beta_{j1}S_{1}+....+\beta_{js}S_{s}$$
Then N (script), the stoichiometric matrix was introduced, as the amount of particles produced by single reactions. Taking the \underline{mass-action principle} we can associate a dynamical system to any reaction network.\\

We have: $$\frac{d}{dt}x = Nv(k,x)$$ $$v_{j} = v_{j}(k_{j},x) = k_{j}\prod_{i=1}^{s} x^{\alpha_{ji}}$$
This dynamical system always has a polynomial RHS.\\

\underline{Example}: MAPk\\
$$E+S_{0} 	\Leftrightarrow ES_{0} \rightarrow E+S_{1} \Leftrightarrow ES_{1} \rightarrow E+S_{2}$$
$$F+S_{2} \Leftrightarrow FS_{2} \rightarrow F+S_{1} \Leftrightarrow FS_{1} \rightarrow F+S_{0}$$
The resulting equations are as follows: $$
\frac{d}{dt}x_{1} = -k_{1}x_{1}x_{1} + (k_{3}-k_{2})x_{6} -k_{5}x_{1}x_{4}+(k_{6}-k_{4})x_{7}
$$ The remainder is left as an exercise.\\

\underline{Remark about Hypergraphs}: Consider again the complex species graph of the MAPk example:

\vspace{5 cm}

\underline{Interaction Graph}: The Interaction graph is ususally denoted by $\stackrel{\rightarrow}{G_{I}} = (V,\stackrel{\rightarrow}{E})$, with $V \equiv S_{1}$ and  $\stackrel{\rightarrow}{E}$ is defined by the Jacobian matrix of the vector field f of the equation $\frac{d}{dt} x = f(x)$, with $x \in \mathbb{R}_{+}^{s}$.\\

First of all, we fix a state $x \in \mathbb{R}_{+}^{s}$ We weight $\stackrel{\rightarrow}{G_{I}}$ on the edges with labels either + , - or 0, depending on whether the Jacobian $(Jf(x))_{ij}$ is positive, negative or 0.\\

\underline{Idea}: given state X, is there an active feedback of variable $x_{i}$ on variable $x_{j}$?\\

For the MAPk example, we have:

\vspace{5 cm}

\underline{Exercise}: Verify this graph.\\ 

We can see higher order feedback loops in this graph- cycles that are either all positive or all negative. There is for example a cycle of length 4: $F\rightarrow FS_{2} \rightarrow S_{1} \rightarrow FS_{1} \rightarrow F$.\\

What can we do with this? For example, we have the \underline{Thomas-Soule Theorem}: (accent e)\\
If a system has no positive cycles, in $\stackrel{\rightarrow}{G_{I}}(x)$ for any x, then it cannot exhibit multi-stationarity.

\end{document}